\documentclass{article}
\usepackage[T1]{fontenc}
\usepackage[utf8]{inputenc}
\usepackage[francais]{babel}

\title{INF2}
\author{Jérôme Arn}
\date{19.02.2019}

\begin{document}

\maketitle
\newpage

\tableofcontents
\newpage
\section{Les classes}
les classes sont des types complexe qui permettent de créer des instances. Les fonctions membres sont des méthodes. Elles sont constituées de deux parties: une partie public et une partie privée. Par défaut, si une déclaration est fait avant la partie privée, elle est considérée comme privée. 
\section{Initialisation des membres}
L'initialisation par liste permet d'éviter d'avoir l'appel du constructeur vide ET l'appel du constructeur que nous avons déclarer. Car en effet lors de l'initialisation par affectation, le constructeur sans argument est appelé et ensuite le constructeur, que nous avons déclaré, est appelé. cela implique de redéclaré le constructeur vid e par défaut. 

\end{document}
