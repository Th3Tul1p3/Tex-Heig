\documentclass{article}
\usepackage[T1]{fontenc}
\usepackage[utf8]{inputenc}
\usepackage[francais]{babel}
\usepackage[top=2cm, bottom=2cm, left=2cm, right=2cm]{geometry}

\title{TIB}
\author{Jérôme Arn}
\date{2 février 2019}

\begin{document}
\section{Introduction}
la téléinformation traite des Services et application, des Protocoles de communication et de l'infrastructure réseau. Des compétences sont transversales comme la sécurité et le développement logicies. Il y a trois principaux réeaux, l'internet qui contient l'internet des objets, les réseaux mobiles avec des réseaux personnels et les réseaux d'entreprises. 
\begin{itemize}
	\item[Marconi :] père de la transmission radio longue distance.
	\item[Claude Shannon :] père de la théorie de l'information, base de la cryptographie. 
	\item[Paul Baran :]inventeur de la communication par paquets.
	\item[Leonard Kleinrock] théorie mathématique des réseaux à paquets, contributeur à l'ARPANET.
	\item[Vint Cerf :]Un des pères d'internet, promotion ipv6.
	\item[Tim Berners-Lee :]inventeur du web, adressage web, HTTP.
\end{itemize}

\section{Réseaux}
\bigskip
\begin{itemize}
	\item[PAN personnal area network:] bluetooth nfc, interconnexion entre périphérique. va de 1 à 10 m.
	\item[LAN Local area network:] réseau d'une entreprise ou d'un campus. ethernet et WIFI
	\item[MAN Metropole area network:]Réseau d'un opérateur dans une ville. fibres optiques, WiMax et Metro-Ethernet. très grande vitesse.  
	\item[WAN wide area network:]Fibres optiques, Faisceaux hertziens et satellites. 10 à 1000km
	\item[internet] interconnexions de tous les réseaux. 
\end{itemize}
\bigskip
Topoligie en Bus, Tous les noeuds sont connectés au même médium. Une seule transmission possible au même moment. distance de communication basse. limité à 1024. Le hub résout le problème, si une machine pose problème sur le bus, elle est éjecter du bus. 
Topologie en anneau, un segement de câble connecte deux stations. résilience contre les pannes. tokenring utilisait un système de jeton pour savoir qui transmet. 
Topologie en étoile, chaque station est connecté à un noeud central. Le noeud central transmet à toutes les stations. 

Topologie maillée, interconnexion entre tous les noeuds. Bonne résilience utilisation 
\section{Commutation}
La commutation est le processus d'acheminement de données à travers un réseau. Il y a deux types de commutation: par circuit et par paquets. 
\subsection{Commutation par circuit}
Utilisés dans les réseaux téléphoniques. Mais cela n'est pas adapté aux réseaux informatiques.
 Car cela n'est pas assez rapide.
\subsection{Commutation par paquets}
La source segmente le message à transmettre en paquets. Ensuite les paquets sont transmis de manières indépendante. Le destinataire trcombine les paquets reçus pour obtenir le message. Il y a plusieurs chemins. Si un paquet est perdu, on retransmet que le paquet. Conversion des formats possible et reroutage facile en cas de panne d'un lien. Mais le délai de transfert est variable et plus long. Pertes de paquets possibles. Noeuds intermédiaires doivent recevoir l'ensemble du paquet pour commencer à le transmettre. Si la file d'attente est trop longue, il se peut qu'on doive recommencer la transmission. 
\subsection{Paramètres de performance}
Délai, Vitesse, Longueur deonnées, Débit binaire et taux de perte.

\section{Modèle de référence}
Les logiciels réseaux sont organisés en pile de couches. Chacunes de ces couches offre un service bien défini. Des entités paires qui communiquent sont du même niveau. Pour communiquer entre elles utilisent un protocole de communication. 
Le chemin réel emprunté par les données traverse toutes les couches de la plus hautes à la plus basse jusqu'au médium physique. L'entité de chaque couche utilise un protocole pour communiquer avec son pair. Chaque protocole peut ajouter un en-tête ou en-queue. Mais ces informations ne sont utilisés que par son entité pair et sont enlevées lors de la lecture par l'entité pair. Les systèmes terminaux, sont les seuls à implémenter toutes les couches. L'implémentation des couches dépend des noeuds intermédiaires. 

\subsection{Référence OSI}
Le modèle OSI a été introduit 1983. Il sert à analyser et concevoir des logiciels réseau. Il définit un cadre pour l'élaboration de chaque couche. 

1.2: des logiciels et client web
\end{document

