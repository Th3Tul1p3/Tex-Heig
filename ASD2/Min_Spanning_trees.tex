% Options for packages loaded elsewhere
\PassOptionsToPackage{unicode}{hyperref}
\PassOptionsToPackage{hyphens}{url}
%
\documentclass[
]{article}
\usepackage{lmodern}
\usepackage[top=1cm, bottom=2cm, left=2cm, right=2cm]{geometry}
\usepackage{amssymb,amsmath}
\usepackage{ifxetex,ifluatex}
\ifnum 0\ifxetex 1\fi\ifluatex 1\fi=0 % if pdftex
  \usepackage[T1]{fontenc}
  \usepackage[utf8]{inputenc}
  \usepackage{textcomp} % provide euro and other symbols
\else % if luatex or xetex
  \usepackage{unicode-math}
  \defaultfontfeatures{Scale=MatchLowercase}
  \defaultfontfeatures[\rmfamily]{Ligatures=TeX,Scale=1}
\fi
% Use upquote if available, for straight quotes in verbatim environments
\IfFileExists{upquote.sty}{\usepackage{upquote}}{}
\IfFileExists{microtype.sty}{% use microtype if available
  \usepackage[]{microtype}
  \UseMicrotypeSet[protrusion]{basicmath} % disable protrusion for tt fonts
}{}
\makeatletter
\@ifundefined{KOMAClassName}{% if non-KOMA class
  \IfFileExists{parskip.sty}{%
    \usepackage{parskip}
  }{% else
    \setlength{\parindent}{0pt}
    \setlength{\parskip}{6pt plus 2pt minus 1pt}}
}{% if KOMA class
  \KOMAoptions{parskip=half}}
\makeatother
\usepackage{xcolor}
\IfFileExists{xurl.sty}{\usepackage{xurl}}{} % add URL line breaks if available
\IfFileExists{bookmark.sty}{\usepackage{bookmark}}{\usepackage{hyperref}}
\hypersetup{
  hidelinks,
  pdfcreator={LaTeX via pandoc}}
\urlstyle{same} % disable monospaced font for URLs
\usepackage{graphicx,grffile}
\makeatletter
\def\maxwidth{\ifdim\Gin@nat@width>\linewidth\linewidth\else\Gin@nat@width\fi}
\def\maxheight{\ifdim\Gin@nat@height>\textheight\textheight\else\Gin@nat@height\fi}
\makeatother
% Scale images if necessary, so that they will not overflow the page
% margins by default, and it is still possible to overwrite the defaults
% using explicit options in \includegraphics[width, height, ...]{}
\setkeys{Gin}{width=\maxwidth,height=\maxheight,keepaspectratio}
% Set default figure placement to htbp
\makeatletter
\def\fps@figure{htbp}
\makeatother
\setlength{\emergencystretch}{3em} % prevent overfull lines
\providecommand{\tightlist}{%
  \setlength{\itemsep}{0pt}\setlength{\parskip}{0pt}}
\setcounter{secnumdepth}{-\maxdimen} % remove section numbering

\date{}

\begin{document}

\hypertarget{header-n2}{%
\section{ASD2}\label{header-n2}}

\begin{itemize}
\item
  la taille c'est le nombre d'arêtes
\item
  l'ordre est le nombre de sommet
\item
  graphe simple : pas de boucle et pas plus d'une arête par paire de
  sommet
\item
  graphe partiel: on supprime quelques arêtes 
\item
  sous graphe: on supprime quelques sommets
\item
  des sommets sont \textbf{adjacents} s'il y a une arête entre eux
\item
  Graphe complet si tous les commets sont connecté entre eux
\end{itemize}

\hypertarget{header-n19}{%
\subsection{Composantes connexes }\label{header-n19}}

La relation de connectivité est réflexive, transitive et symétrique.

\hypertarget{header-n21}{%
\subsection{Cycles d'Euler et Hamilton}\label{header-n21}}

Un graphe connexe admet un cycle Eulérien si et seulement si il n'a pas
de sommet de degré impair. Le premier et le dernier sommet peuvent être
di érents, et dans ce cas de degré impair.

Un cycle Eulérien passe une et une seule fois par chaque arrête et un
cycle Hamiltonien passe une et une seule fois par chaque sommet.

Chaîne Eulérienne: Le premier et le dernier sommet peuvent être di
érents, et dans ce cas de degré impair.

\hypertarget{header-n23}{%
\section{Expressions Lambdas }\label{header-n23}}

\begin{figure}
\centering
\includegraphics{/media/Partage/HEIG/Tex-Heig/ASD2/img/LAMBDA.png}
\caption{}
\end{figure}

\hypertarget{header-n25}{%
\section{Graphe orienté}\label{header-n25}}

Représentation par matrice d'adjacence \textbf{O(V)} efficace seulement
si le graphe est complet. Par listes d'adjacences \textbf{O(degré(v))}.

\textbf{Parcours avec un DFS pour savoir quels sont les sommets
atteignables depuis un sommet. } \textbf{Pour savoir le plus court
chemin il faut faire un BFS}

\begin{itemize}
\item
  sommet demi-degré intérieur : flèches entrantes
\item
  sommet demi-degré extérieur : flèches sortantes
\item
  Tri topologique : trier les sommets pour que les arcs pointent vers le
  haut 
\item
  connectivité forte : chemin entre tous les couples de sommets La
  matrice n'est pas symétrique. 
\item
  composantes fortement connexe: est un ensemble de sommet fortement
  connectés 
\item
  il y a deux types d'arcs:

  \begin{itemize}
  \item
    ceux qui restent dans la même CFC 
  \item
    Ceux qui mènent d'une CFC à une autre 
  \end{itemize}
\item
  Si on regroupe tous les sommets d'une même CFC on obtient un graphe
  acyclique 
\end{itemize}

\hypertarget{header-n48}{%
\subsection{Tri topologique }\label{header-n48}}

Trier les sommets de manière à ce que tous les sommets pointent dans la
même direction. N'est possible que pour un graphe acyclique. Se trouve
avec un DFS post-ordre inversé.

Pour détecter un cycle on utilise la pile récursion. si dans le parcours
on tombe sur un sommet se trouvant déjà dans la pile -\/-\textgreater{}
cycle.

\hypertarget{header-n51}{%
\subsection{Connectivité forte }\label{header-n51}}

Deux sommets sont fortement connectés s'il existe un chemin dans un sens
et dans l'autre. C'est une relation d'équivalence si elle est réflexive,
symétrique et transitive. Une CFC est un ensemble maximal de sommets
fortement connectés.

On distingue deux types d'arcs:

\begin{itemize}
\item
  ceux qui restent dans la même CFC, Si on regroupe tous les sommets
  d'une même CFC on obtient un graphe acyclique. De cette manière on
  peut le trier topologiquement. 
\item
  ceux qui mènent dans une CFC à l'autre
\end{itemize}

Un graphe G et son graphe inverse ont les mêmes composantes fortement
connexes. \textbf{L'algorithme de Kosaraju-Sharir calcul le post ordre
inverse du parcours en profondeur d'un graphe inverse.}

\hypertarget{header-n60}{%
\subsection{Algorithme de kosaraju-sharir }\label{header-n60}}

Calcul du post-ordre inverse DFS sur le graphe inverse.

\hypertarget{header-n62}{%
\section{Union find }\label{header-n62}}

\hypertarget{header-n63}{%
\subsection{Recalcul des composantes connexes}\label{header-n63}}

Lorsque le graphe n'a aucune arrête, chaque sommet forme une CC
différente qu'on étiquette par le numéro du sommet. Chaque sommet pointe
sur une sommet représentatif

\begin{itemize}
\item
  quick find pour savoir la CC d'un sommet \textbf{O(1)}
\item
  connected savoir si deux sommets sont de la même CC
\item
  union unir deux CC \textbf{O(V)}
\end{itemize}

Faire pointer chaque sommet sur un autre sommet de la CC et identifier
chaque sommet par un sommet représentatif qui pointe sur lui-même.

complexité en log*(n) augmente très lentement

\hypertarget{header-n74}{%
\section{Minimum spanning Trees}\label{header-n74}}

Sur un graphe dont les arrêtes ont des poids positifs. Un arbre couvrant
est un sous-graphe qui est \textbf{Connexe, acyclique et qui inclus tous
les sommets}. Trouver un arbre couvrant tous les sommets d'un poids
minimum.

\begin{itemize}
\item
  sans cycle et avec n-1 arrêtes
\item
  connexe et admet n-1 arrêtes 
\item
  sans cycle, en ajoutant une arrête on crée un et un seul cycle
  élémentaire
\item
  connexe, en supprimant une arrête quelconque, il n'est plus connexe. 
\item
  il existe une chaîne et une seule entre 2 sommets. 
\end{itemize}

\hypertarget{header-n87}{%
\subsection{Algorithme glouton}\label{header-n87}}

Un algorithme qui suit le principe de faire, étape par étape, un choix
optimum local.

La coupe d'un graphe est une partition en deux sous ensemble disjoints
et exhaustifs. C'est aussi un ensemble des arêtes ayant un sommet dans
chacune des deux partitions.

\begin{itemize}
\item
  On choisi une coupe sans arrête déjà dans le MST recommencer jusqu'à
  que V-1 arrêtes soient rouges. 
\end{itemize}

\hypertarget{header-n93}{%
\subsection{Algorithme de Kruskal}\label{header-n93}}

Fait croître jusqu'à couverture totale. On considère les arrêtes par
liste de poids croissant. si l'arrête ne fait pas de cycle, on l'ajoute.
On teste si l'arrête ajoutée crée un cycle avec des union-find avec une
complexité de O(log*V) qui augmente très lentement. Quand on ajoute une
arrête avec les sommets v et w on appel union(v,w) si ils ne sont pas
connected(v,w). Puis on utilise une priority queue pour trier les
arrêtes.

Le calcul du MST se fait en temps proportionnel de E * log(E) mais si
les arrêtes sont déjà ordonné, cela ne prend que E * log(V) qui est
linéaire en E.

\hypertarget{header-n96}{%
\subsection{Algorithme de Prim}\label{header-n96}}

\hypertarget{header-n97}{%
\subsubsection{Version paresseuse}\label{header-n97}}

\begin{itemize}
\item
  On fait une PQ avec toutes les arrêtes connectées à l'arbre courant
  par au moins un sommet. 
\end{itemize}

\hypertarget{header-n101}{%
\subsubsection{Version stricte}\label{header-n101}}

\begin{itemize}
\item
  On fait une PQ avec les sommets, la priorité de ces derniers est le
  poids le plus petit parmi les arêtes connectant à ce sommet à l'arbre.
\end{itemize}

Quand on sort un sommet v de la PQ, on ajoute l'arête v-w associée à
l'arbre et l'on considère les\\
arêtes w-x adjacentes. Si x est dans l'arbre, on l'ignore Si x est
absent de la PQ, on l'ajoute avec la priorité du poids de w-x Sinon, on
abaisse la priorité de x si le poids de w-x est plus bas que le poids
dans la PQ

\hypertarget{header-n105}{%
\section{Chemin les plus courts }\label{header-n105}}

Terminologie : raccourcir le chemin = "relâcher" un ressort imaginaire
qui va du sommet source à w, parce qu'on a trouvé un chemin plus court
qu'auparavant

\hypertarget{header-n107}{%
\subsection{Graphes acyclique }\label{header-n107}}

Traiter les sommets par ordre topologique.

\hypertarget{header-n109}{%
\subsection{Algorithme de Dijkstra}\label{header-n109}}

On traite une seule fois chaque sommet, on traite les sommets par ordre
de distance croissante. \textbf{Seulement pour les poids non-négatifs}.
O(E*log(V))

\hypertarget{header-n111}{%
\subsection{Algorithme de Bellman-Ford}\label{header-n111}}

On s'arrête à V-1 passages. Faire attention au circuit absorbant car on
peut avoir des poids négatifs. Si après V-1 passage on peut toujours
diminuer une distance entre deux sommets, il y a un circuit absorbant.
On relâché1 plusieurs fois chaque sommet. O(E*V)

\hypertarget{header-n113}{%
\section{Complexité }\label{header-n113}}

Le parcours en largeur sur matrice d'adjacence O(V*V) et par liste
d'adjacence O(V+E).

Méthode reverse: O(N). insertion set O(log(N)),

\begin{figure}
\centering
\includegraphics{/media/Partage/HEIG/Tex-Heig/ASD2/img/Graphe.png}
\caption{}
\end{figure}

\begin{figure}
\centering
\includegraphics{/media/Partage/HEIG/Tex-Heig/ASD2/img/kruskal.png}
\caption{}
\end{figure}

\hypertarget{header-n116}{%
\subsection{Dijkstra}\label{header-n116}}

\begin{figure}
\centering
\includegraphics{/media/Partage/HEIG/Tex-Heig/ASD2/img/Dijskstra.png}
\caption{}
\end{figure}

\hypertarget{header-n126}{%
\section{Quiz}\label{header-n126}}

Pour un graphe creux, il est préférable de choisir une implémentation
par listes d'adjacence.

\end{document}
