\documentclass{article}
\usepackage[T1]{fontenc}
\usepackage[utf8]{inputenc}
\usepackage[francais]{babel}
\usepackage{ulem}

\title{ASD1}
\author{jerome Arn}
\date{19.02.2019}

\begin{document}

\maketitle
\newpage

\tableofcontents
\newpage

\section{Qualité d'un programme}
La fiabilité, la robustess, l'extensibilité, la réutilisabilité, la compatibilité et l'efficacité.

\uline{Algorithme:} ensemble fini d'étape dont le but est de résoudre un problème ou d'accomplir une tâche. Ces étapes sont forméesd'un ensemble fini d'opérations. 

\newline Un problème computationnel est spécifié de façon abstraite par :
\bigskip
\begin{itemize}
	\item[I:] Ensemble des entrées 
	\item[O:] Ensemble des sorties
    	\item[R:] Dépendance relationnelle. la sortie en fonction de toutes les entrées possibles
\end{itemize}
\bigskip

\section{Complexité}
\begin{itemize}
	\item[Une incrémentation:]Complexité de O(1)
	\item[une boucle fait N fois:] Complexité de O(N)
    	\item[des boucles imbriquées:] complexité quadratique 
	\item[Boucles avec *n ou /n:] sont de complexité logarithmique
	\item[enchaînement alternatif:] La complexité dans le meilleur des cas, dans le pire des cas et en moyenne
\end{itemize}
\bigskip

\newline 
\end{document}
