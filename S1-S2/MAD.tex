\documentclass{article}
\usepackage[T1]{fontenc}
\usepackage[utf8]{inputenc}
\usepackage[francais]{babel}
\usepackage[top=2cm, bottom=2cm, left=2cm, right=2cm]{geometry}

\title{Math TIC}
\author{Jérôme Arn}
\date{24 septembre 2019}

\begin{document}

\maketitle
\newpage

\tableofcontents
\newpage

\section{Les ensembles}
Un ensemble est constitué de un ou plusieurs élements. Dans le premier cas, 

\section{Chapitre 5}

\subsection{Permutations}
Une permutation de n objets distincts correspon à un ordre particulier de ces n objets. 

Pour une permutation à n objets, la formule devient n! 

\subsection{Arrangements}
Un arrangement de k objets parmi n est une liste ordonnée de k objets distincts, choisis parmi n objets distincts. Une permutation est un arrangement de n objet parmi n.  

$$ n!/(n - k)!$$ 

\subsection{Combinaisons}
Une combinaison de k objets parmi n est un sous-ensemble de k objets distincts, choisis parmi n objets distincts. Oû l'ordre n'a aucune importance. 

$$ n!/k!(n-k)!$$ 

\subsection{Binôme de Newton}


\end{document}
