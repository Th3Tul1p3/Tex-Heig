\documentclass{article}
\usepackage[T1]{fontenc}
\usepackage[utf8]{inputenc}
\usepackage[francais]{babel}
\usepackage{ulem}

\title{ASD1}
\author{jerome Arn}
\date{19.02.2019}

\begin{document}

\maketitle
\newpage

\tableofcontents
\newpage

\section{Qualit� d'un programme}
La fiabilit�, la robustess, l'extensibilit�, la r�utilisabilit�, la compatibilit� et l'efficacit�.

\uline{Algorithme:} ensemble fini d'�tape dont le but est de r�soudre un probl�me ou d'accomplir une t�che. Ces �tapes sont form�esd'un ensemble fini d'op�rations. 

\newline Un probl�me computationnel est sp�cifi� de fa�on abstraite par :
\bigskip
\begin{itemize}
	\item[I:] Ensemble des entr�es 
	\item[O:] Ensemble des sorties
    	\item[R:] D�pendance relationnelle. la sortie en fonction de toutes les entr�es possibles
\end{itemize}
\bigskip

\section{Complexit�}
\begin{itemize}
	\item[Une incr�mentation:]Complexit� de O(1)
	\item[une boucle fait N fois:] Complexit� de O(N)
    	\item[:]
\end{itemize}
\bigskip

\end{document}
