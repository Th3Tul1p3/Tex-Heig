\documentclass{article}
\usepackage[T1]{fontenc}
\usepackage[utf8]{inputenc}
\usepackage[francais]{babel}
\usepackage[top=2cm, bottom=2cm, left=2cm, right=2cm]{geometry}
\usepackage{amsmath}
\usepackage{textcomp}
\usepackage{amssymb}

\title{Math TIC}
\author{Jérôme Arn}
\date{24 septembre 2018}

\begin{document}

\maketitle
\newpage

\tableofcontents
\newpage

\section{trigonométrie}

\[\frac{\alpha^{rd}}{\pi} = \frac{\alpha^{\degres}}{180}\]
$$ L = \alpha^{rd}*r$$

\section{calcul matricielle}
\[A=[a_ij] \quad ssi \quad 1 \leqslant i \leqslant m \quad et \quad  1 \leqslant j \leqslant n \quad M_{mxn}(\mathbb{R})\]

\[A=[(-1)^{i+j}] \quad ssi \quad 1 \leqslant i \leqslant 2 \quad et \quad  1 \leqslant j \leqslant 3 \quad M_{mxn}(\mathbb{R})\]

\[
M=
  \begin{pmatrix}
    a_11 & a_12 & a_13  \\
    a_21 & a_22 & a_23
  \end{pmatrix}
  =
  \begin{pmatrix}
    (-1)^{1+1} & (-1)^{1+2} & (-1)^{1+3}  \\
    (-1)^{2+1} & (-1)^{2+1} & (-1)^{2+1}
  \end{pmatrix}
\]


une matrice A est un tableau de m lignes et n colonne : A appartient à M n*M
si m=n A est une matrice carrée. 

\subsection{Egalité}
\[A,B \in M_{mxn}(\mathbb{R}) \quad A = B \Longleftrightarrow a_ij = b_ij\]
\[A = [a_ij], B = [b_ij]\]

\subsection{Addition}
\[A,B \in M_{mxn}(\mathbb{R}) \quad A = [a_ij] \quad B = [b_ij]\]
\[C = A + B \in M_{mxn}(\mathbb{R}) \quad C = [c_ij]\]
\[c_ij = a_ij + b_ij\]

Exemple:

\[
A=
  \begin{pmatrix}
    2 & 3 \\
    2 & 4
  \end{pmatrix}
B =
  \begin{pmatrix}
    1 & 2  \\
    2 & 1
  \end{pmatrix}
\]

\[
C = A + B =
  \begin{pmatrix}
    2 & 3 \\
    2 & 4
  \end{pmatrix}
+
  \begin{pmatrix}
    1 & 2  \\
    2 & 1
  \end{pmatrix}
  =
  \begin{pmatrix}
    3 & 5  \\
    4 & 5
  \end{pmatrix}
\]

\subsection{Multiplication}
\[A \in M_{mxn}(\mathbb{R}) \quad A = [a_ij] \quad \lambda \in (\mathbb{R}) \quad \lambda A = [\lambda a_ij] \]

Exemple:

\[
3
  \begin{pmatrix}
    2 & 3 \\
    2 & 4
  \end{pmatrix}
  =
  \begin{pmatrix}
    6 & 9  \\
    6 & 12
  \end{pmatrix}
\]
\subsection{Multiplication matricielle}
Condition: nombre de colonne A 0 nombre de ligne B.

\[ AxB \quad A \in M_{mxn}(\mathbb{R}) \quad M_{nxp}(\mathbb{R})\]

\[
A=
  \begin{pmatrix}
    1 & 2 & 3 \\
    -1 & 3 & 4
  \end{pmatrix}
B =
  \begin{pmatrix}
    2 & 4 \\
    1 & 5  \\
		-2 & -1
  \end{pmatrix}
\]

\[
AxB=
  \begin{pmatrix}
    1*2+2*1+3*-2 & 1*4+2*5+3*-1 \\
    -1*2+3*1+4*-2 & -1*4+3*5+4*-1
  \end{pmatrix}
=
  \begin{pmatrix}
    2 & 4\\
    -2 & 11  \\
		-7 & 7
  \end{pmatrix}
\]

Remarques: le produit matricielle n'est pas commutatif.

\subsection{Propriétés}
\[A( B + C ) = AB + AC\]
\[(\lambda + \mu) A = A \lambda + A \mu\]
\[(\lambda \mu) A = \lambda (\mu A)\]

\section{Matrices Particulières}
\subsection{Matrice diagonale}
\[A \in M_{nxn}(\mathbb{R}) = M_{n}(\mathbb{R})\]

\[
A=
  \begin{pmatrix}
    2 & 0 & 0 \\
    0 & 2 & 0 \\
    0 & 0 & 2
  \end{pmatrix}
\]

\subsection{Matrice triangulaire supérieur}
\[A \in M_{n}(\mathbb{R})\]

\[
A=
  \begin{pmatrix}
    1 & 2 & 3 \\
    0 & 1 & 4 \\
    0 & 0 & 1
  \end{pmatrix}
\]

\subsection{Matrice triangulaire inférieur}
\[A \in M_{n}(\mathbb{R})\]

\[
A=
  \begin{pmatrix}
    1 & 0 & 0 \\
    4 & 1 & 0 \\
    3 & 2 & 1
  \end{pmatrix}
\]

\subsection{Matrice identité}
\[A \in M_{n}(\mathbb{R}) \quad I_n A = A I_n = A\]

\[
I_n=
  \begin{pmatrix}
    1 & 0 & 0 \\
    0 & 1 & 0 \\
    0 & 0 & 1
  \end{pmatrix}
\]


\subsection{Matrice nulle}
\[A \in M_{n}(\mathbb{R}) \quad 0_n + A = A + 0_n = A\]

\[
O_n=
  \begin{pmatrix}
    0 & 0 & 0 \\
    0 & 0 & 0 \\
    0 & 0 & 0
  \end{pmatrix}
\]

\subsection{Transposer des matrices}
\[A \in M_{n}(\mathbb{R}) \quad A^t = \text{transposé de } A \]
les lignes de A sont les colonnes de A transposé.

\[A \in M_{mxn}(\mathbb{R}) \quad A^t \in M_{nxm}(\mathbb{R})\]

\[
A=
  \begin{pmatrix}
    1 & 2 & 3 \\
    1 & 2 & 3 \\
    1 & 2 & 3
  \end{pmatrix}
= A^t
  \begin{pmatrix}
    1 & 1 & 1 \\
    2 & 2 & 2 \\
    3 & 3 & 3
  \end{pmatrix}
\]

\subsubsection{Propriétés}
\[(A^t)^t = A \quad (A + B)^t = A^t + B^t \quad (AB)^t = (B^t A^t)\]

\subsubsection{Matrice symétrique}
\[A \in M_{n}(\mathbb{R}) \quad A^t = A\]

\subsubsection{Matrice anti-symétrique}
\[A \in M_{n}(\mathbb{R}) \quad A^t = -A\]

\end{document}