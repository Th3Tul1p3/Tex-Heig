\documentclass[a4paper,11pt]{report}
\usepackage[T1]{fontenc}
\usepackage[utf8]{inputenc}
\usepackage{lmodern}
\usepackage[francais]{babel}


\usepackage{amsmath}
\usepackage{amsfonts}
\usepackage{amssymb}

% Start U+1D400 // Symbols math alphanum
% Start U+ 2100 // Symbols de type lettre
\DeclareUnicodeCharacter{1D538}{\mathbb{A}}
\DeclareUnicodeCharacter{1D539}{\mathbb{B}}
\DeclareUnicodeCharacter{2102}{\mathbb{C}}
\DeclareUnicodeCharacter{1D53B}{\mathbb{D}}
\DeclareUnicodeCharacter{1D53C}{\mathbb{E}}
\DeclareUnicodeCharacter{1D53D}{\mathbb{F}}
\DeclareUnicodeCharacter{1D53E}{\mathbb{G}}
\DeclareUnicodeCharacter{210D}{\mathbb{H}}
\DeclareUnicodeCharacter{1D540}{\mathbb{I}}
\DeclareUnicodeCharacter{1D541}{\mathbb{J}}
\DeclareUnicodeCharacter{1D542}{\mathbb{K}}
\DeclareUnicodeCharacter{1D543}{\mathbb{L}}
\DeclareUnicodeCharacter{1D544}{\mathbb{M}}
\DeclareUnicodeCharacter{2115}{\mathbb{N}}
\DeclareUnicodeCharacter{1D546}{\mathbb{O}}
\DeclareUnicodeCharacter{2119}{\mathbb{P}}
\DeclareUnicodeCharacter{211A}{\mathbb{Q}}
\DeclareUnicodeCharacter{211D}{\mathbb{R}}
\DeclareUnicodeCharacter{1D54A}{\mathbb{S}}
\DeclareUnicodeCharacter{1D54B}{\mathbb{T}}
\DeclareUnicodeCharacter{1D54C}{\mathbb{U}}
\DeclareUnicodeCharacter{1D54D}{\mathbb{V}}
\DeclareUnicodeCharacter{1D54E}{\mathbb{W}}
\DeclareUnicodeCharacter{1D54F}{\mathbb{X}}
\DeclareUnicodeCharacter{1D550}{\mathbb{Y}}
\DeclareUnicodeCharacter{2124}{\mathbb{Z}}

\DeclareUnicodeCharacter{1D552}{\mathbb{a}}
\DeclareUnicodeCharacter{1D553}{\mathbb{b}}
\DeclareUnicodeCharacter{1D554}{\mathbb{c}}
\DeclareUnicodeCharacter{1D555}{\mathbb{d}}
\DeclareUnicodeCharacter{1D556}{\mathbb{e}}
\DeclareUnicodeCharacter{1D557}{\mathbb{f}}
\DeclareUnicodeCharacter{1D558}{\mathbb{g}}
\DeclareUnicodeCharacter{1D559}{\mathbb{h}}
\DeclareUnicodeCharacter{1D55A}{\mathbb{i}}
\DeclareUnicodeCharacter{1D55B}{\mathbb{j}}
\DeclareUnicodeCharacter{1D55C}{\mathbb{k}}
\DeclareUnicodeCharacter{1D55D}{\mathbb{l}}
\DeclareUnicodeCharacter{1D55E}{\mathbb{m}}
\DeclareUnicodeCharacter{1D55F}{\mathbb{n}}
\DeclareUnicodeCharacter{1D560}{\mathbb{o}}
\DeclareUnicodeCharacter{1D561}{\mathbb{p}}
\DeclareUnicodeCharacter{1D562}{\mathbb{q}}
\DeclareUnicodeCharacter{1D563}{\mathbb{r}}
\DeclareUnicodeCharacter{1D564}{\mathbb{s}}
\DeclareUnicodeCharacter{1D565}{\mathbb{t}}
\DeclareUnicodeCharacter{1D566}{\mathbb{u}}
\DeclareUnicodeCharacter{1D567}{\mathbb{v}}
\DeclareUnicodeCharacter{1D568}{\mathbb{w}}
\DeclareUnicodeCharacter{1D569}{\mathbb{x}}
\DeclareUnicodeCharacter{1D56A}{\mathbb{y}}
\DeclareUnicodeCharacter{1D56B}{\mathbb{z}}


% Start U+2200 // Opérteur mathd
\DeclareUnicodeCharacter{2260}{\neq}

\DeclareUnicodeCharacter{2208}{\in}
\DeclareUnicodeCharacter{2209}{\notin}

\DeclareUnicodeCharacter{2200}{\forall}
\DeclareUnicodeCharacter{2203}{\exists}
\DeclareUnicodeCharacter{2204}{\nexists}
\DeclareUnicodeCharacter{2282}{\subset    }
\DeclareUnicodeCharacter{2283}{\supset    }
\DeclareUnicodeCharacter{2284}{\not\subset   }
\DeclareUnicodeCharacter{2285}{\not\supset   }
\DeclareUnicodeCharacter{2286}{\subseteq  }
\DeclareUnicodeCharacter{2287}{\supseteq  }
\DeclareUnicodeCharacter{2288}{\nsubseteq }
\DeclareUnicodeCharacter{2289}{\nsupseteq }


%\usepackage{boisik}

\title{Analyse \\ Note de cours}
\author{Gabriel Roch}
\date{19 février 2019 -- \today}

\begin{document}

\maketitle
\tableofcontents

\part{Note de cours}
\chapter{Nombre complexes}

\begin{align*}
   ℕ &= \{0,1,2,3,4,…\} & -1 ∉ ℕ \\
   ℤ &= \{…, -2, -1, 0, 1, 2, …\} & \frac{1}{2} ∉ ℤ \\
   ℚ &= \{\frac{m}{n} | m ∈ ℤ, n ∈ ℕ*\} & \pi, e ∉ ℚ \\
   ℝ &= \{\frac{m}{n}, -2, 0, 3, e, \pi, …\} \\
   ℂ &= \text{Nombre complexes} & ∈ ℝ^2 \\
   ℍ &= \text{Coprs des quaternions (sans commutativité)} & ∈ ℝ^4
\end{align*}
\begin{align*}
   j^2 &= -1 \\
   Z &= a + jb & a,b &∈ ℝ & \text{Forme rectangulaire} \\
   Z* &= a - jb & \bar{Z} &= Z* & \text{Conjugué de Z}
\end{align*}

%R : droite a ≠ 0 => 1/a exists a·1/b = 1
%
%C : R² = R×R
%
%R, j \notin R
%IMPORTANT: j² = -1
%
%Z \in C
%// forme réctangulaire | carthésienne
%Z = a+jb
%
%Z₁ = a₁+jb₁
%Z₂ = a₂+jb₂
%
%Z₁Z₂ = (a_1+jb_1)(a_2+jb_2) = …
%
%IMPORTANT: Z_1Z_2 = (a_1a_2 - b_1b_2) + j(a_1b_2 + a_2b_1)
%
%Z₁Z₂ = Z₂Z₁
%
%EXAMPLE
%------
%
%z=1j
%
%
%R³ → pas de corps
%R⁴ → Hamilton Quatenions (corps) = H  [mais pas commutatif]  surtout en mécanique
%Corps fini pour la sécu

\section{Conjugaison}
\begin{align}
   z &= a + jb & a,b \in \mathbb{R} \\
   z* &= a-jb
   z z* &= (a+jb)(a-jb) = a^2 - j^2b^2 = a^2 + b^2 \\
   z z* &= d^2(0,z)
\end{align}
Form rectangulaire de $\frac{1}{z}$, $z \neq 0$

\begin{align}
   \frac{1}{z} &= \frac{1 \cdot z*}{z\cdot z*} = \frac{ a-jb}{a^2+b^2} \\
   \frac{1}{z} &= \frac{a}{a^2 + b^2} - j\frac{b}{a^2+b^2} & \text{Frome rect de 1/z}
\end{align}
Exemple

\begin{align}
   \frac{1}{2-j} &= \frac{2+j}{(2-j)(2+j)} = \frac{2+j}{5} = 2/5 + \frac{1}{5} j \\
   z = a+jb\\
   Re(z) &= \frac{z + z*}{2} \\
   Im(z) &= \frac{z-z*}{2} \\
\end{align}

\begin{align}
   Re(z) &= a\\
   z + z* &= a [[+jb]] + a [[ -jb]] = 2a \\
   a = \frac{z+z*}{2} \\
   z-z* = (a+jb)- (a-jb) = 2jb \\
   b = \frac{z-z*}{2j}
\end{align}

\begin{align}
   j^n = j^{n \text{ mod } 4}
\end{align}

\subsection{Propriétés}
\begin{align}
   (Z_1 + Z_2)* &= Z_1* + Z_2* \\
   (Z*)* &= Z \\
   (Z_1Z_2)* &= Z_1* Z_2* \\
   \left(\frac{Z_1}{Z_2}\right)* &= \frac{Z_1*}{Z_2*} \\
\end{align}

\section{Module de $Z$}
\begin{align*}
   Z &= a+jb \\
   |Z| &= \sqrt{a^2+b^2} & a,b ∈ ℝ
   |Z|^2 &= ZZ* \\
   |Z| = \sqrt{ZZ*} \\
\end{align*}
\begin{align*}
   |Z| &\geqslant 0 \\
   |Z| &= d(0,Z) \\
\end{align*}

\subsection{Propriétés}
\begin{align}
   |Z| &\geqslant 0 \\
   Z_1 &= a_1 + b_1j & Z_2 &= a_2 + b_2j & |Z_1Z_2| &= |Z_1||Z_2|  \\
   Z_1 &= a_1 + b_1j & Z_2 &= a_2 + b_2j & |\frac{Z_1}{Z_2}| &= \frac{|Z_1|}{|Z_2|} \\
   |Z|^2 &= ZZ* \\
   |Z| &= \sqrt{ZZ*} \\
\end{align}
\begin{align*}
   ZZ* &= (a+jb)(a-jb) = a^2-(jb)^2 = a^2 + b^2 &&= |Z|^2
\end{align*}

Le module ($|Z|$) est une extension de la valeur absolue.

\[
   |(1+j)^{10}| = |1+j|^{10} = \sqrt{2}^{10} = 2^5 = 32
\]

exemple

\[
   |(\frac{1-j}{1+j})^{10}| = \frac{|1-j|^{10}}{|1+j|^{10}} = \frac{\sqrt{2}^{10}}{\sqrt{2}^{10}} = 1
\]

\end{document}
